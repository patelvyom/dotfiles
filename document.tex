\documentclass[fleqn]{article}
\usepackage{geometry}
\geometry{margin=1in}
\usepackage{amsmath}
\usepackage{amsthm}
\usepackage{amssymb}
\usepackage{amsfonts}
\usepackage{ifthen}
\usepackage{pdfpages}
\usepackage{graphicx} 
\usepackage{setspace}
\usepackage{extarrows} 
\usepackage{mathtools}
\usepackage{microtype}

%\usepackage[sc]{mathpazo}
\linespread{1.2}
\allowdisplaybreaks

%The two lines below essentially have same "feel" of mathpazo but makes the
%math symbols better compared to regular mathpazo math symbols.
\usepackage{newpxmath}
\usepackage{newpxtext}

\DeclarePairedDelimiter{\floor}{\lfloor}{\rfloor}
\DeclarePairedDelimiter{\ceil}{\lceil}{\rceil}
\renewcommand{\bf}[1]{\textbf{#1}}
\renewcommand{\it}[1]{\textit{#1}}

\newcommand{\N}{\mathbb{N}}
\newcommand{\Q}{\mathbb{Q}}
\newcommand{\R}{\mathbb{R}}
\newcommand{\Z}{\mathbb{Z}}
\newcommand{\C}{\mathbb{C}}
\newcommand{\fn}[3]{#1 : #2 \rightarrow #3}
\newcommand{\br}[1]{\left( #1 \right)}
\newcommand{\curly}[1]{\left\{ #1 \right\}}
\newcommand{\set}[2]{\curly{#1\ \textbf{:}\ #2}}
\newcommand{\im}{\textbf{im }}
\newcommand{\codom}{\textbf{codom }}
\newcommand{\sbr}[1]{\left[ #1 \right]}
\newcommand{\eqn}[1]{\begin{eqnarray*} #1 \end{eqnarray*}}
\newcommand{\abs}[1]{\left| #1 \right|}
\newcommand{\eps}{\epsilon}
\newcommand{\del}{\delta}
\newcommand{\limit}[3]{\lim_{#1 \rightarrow #2}#3}
\newcommand{\bmat}[1]{\ensuremath{\begin{bmatrix} #1 \end{bmatrix}}}
\newcommand{\vmat}[1]{\ensuremath{\begin{vmatrix} #1 \end{vmatrix}}}
\newcommand{\smat}[1]{\ensuremath{\left[\begin{smallmatrix} #1 \end{smallmatrix}\right]}}
\newcommand{\diff}[3][]{%                                                                                                          
	\ifthenelse{\equal{#1}{}}{%
		\ensuremath{\frac{d{#2}}{d{#3}}}%
	}{%                                                                                                                                  
		\ensuremath{\frac{d^{#1}\!{#2}}{d{#3}^{#1}}}%                                                                                                   
	}%
}
\title{\textbf{TITLE HERE}
	
	TITLE HERE
}

\author{Vyom Patel \ NSID : vnp614}
\date{\today}

\begin{document}
\maketitle

\subsection{Mathpazo text has been used in this template.}

We denote the set of real numbers as $\R$ and that of complex numbers as
$\C$.

A map is written as $\fn{f}{X}{Y}$ where $\im f = f(X)$ and $\codom f = Y$.

These are large brackets:

\[
\br{\frac{a}{b}}\br{a + b}
\]

A simple set:
\[
  \curly{\frac{1}{2}, \frac{2}{3}, \frac{\pi^2}{6}}
\]

A set with a predicate:
\[
  \set{z}{\zeta\br{z} = 0}
\]

Here is a matrix:
\[
  \bmat{
    1 & 1 \\
    0 & 1 \\
  }
\]

Equivalence class of a sequence:
\[
  \sbr{\frac{1}{n}}
\]

This is an aligned equation:
\eqn{
  A &=& 1\\
  B &=& 2\\
  C &=& 3
}

This is a limit:
\eqn{
  \limit{n}{\infty}{\sum_{i = 1}^n \frac{1}{i^2}}
}

Limits of functions $\fn{f}{\R^m}{\R^n}$ are defined as follows:
\eqn{
  \limit{x}{a}{f(x)} = p \iff
    \forall \eps > 0, \exists \del > 0, \abs{x - a} < \del
    \implies \abs{f(x) - f(a)} < \eps
}

Here are some derivatives:
\eqn{
  \diff{n}{x}{\br{\sum_{k = 1}^{\infty}\frac{1}{k^x}}}
}

\end{document}
